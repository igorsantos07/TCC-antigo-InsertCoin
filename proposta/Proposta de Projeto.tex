%%%%%%%%%%%%%%%%%%%%%%%%%%%%%%%%%%% CABECALHO %%%%%%%%%%%%%%%%%%%%%%%%%%%%%%%%%%%
\documentclass[a4paper]{abnt}

\usepackage[brazil]{babel}
\usepackage[utf8]{inputenc}
\usepackage[T1]{fontenc}

\author{Igor~Santos}
\title{Proposta de Trabalho para Projeto de TCC}

\begin{document}
\maketitle

%%%%%%%%%%%%%%%%%%%%%%%%%%%%%%%%%%%%% INICIO %%%%%%%%%%%%%%%%%%%%%%%%%%%%%%%%%%%%%
\chapter{\mbox{Proposta de Trabalho de Projeto de TCC}}

\textbf{Ano:} \underline{2014} \textbf{Semestre:} \underline{2}

\textbf{Nome do Aluno:} \underline{Igor Gomes dos Santos}

\textbf{Matrícula:} \underline{2011.01.12519-5}

\textbf{Professor Orientador:} \underline{José Carlos Millan}

\textbf{Título Provisório do Projeto:} \underline{Insert Coin}



%%%%%%%%%%%%%%%%%%%%%%%%%%%%%%%%%%%%%%%%%%%%%%%%%%%%%%%%%%%%%%%%%%%%%%%%%%%%%%%%%%%%%%%%%
\section{Introdução}


%%%%%%%%%%%%%%%%%%%%%%%%%%%%%%%%%%%%%%%%%%%%%%%%%%%%%%%%%%%%%%%%%%%%%%%%%%%%%%%%%%%%%%%%%
\section{O Projeto}


%%%%%%%%%%%%%%%%%%%%%%%%%%%%%%%%%%%%%%%%%%%%%%%%%%%%%%%%%%%%%%%%%%%%%%%%%%%%%%%%%%%%%%%%%
\section{Problemas a serem resolvidos}


%%%%%%%%%%%%%%%%%%%%%%%%%%%%%%%%%%%%%%%%%%%%%%%%%%%%%%%%%%%%%%%%%%%%%%%%%%%%%%%%%%%%%%%%%
\section{Vantagens para o usuario}


%%%%%%%%%%%%%%%%%%%%%%%%%%%%%%%%%%%%%%%%%%%%%%%%%%%%%%%%%%%%%%%%%%%%%%%%%%%%%%%%%%%%%%%%%
\section{Aprendizado para o aluno}


%%%%%%%%%%%%%%%%%%%%%%%%%%%%%%%%%%%%%%%%%%%%%%%%%%%%%%%%%%%%%%%%%%%%%%%%%%%%%%%%%%%%%%%%%
\section{Metodologia e Tecnologias a serem utilizadas}
O projeto sera elaborado utilizando as seguintes ferramentas:

\subsection{Para a documentação}
\begin{itemize}
	\item Análise dos requisitos e objetivos do projeto em \LaTeX{} (para portabilidade e flexibilidade de desenvolvimento dos documentos);
	\item Diagramas UML (para fluxos de dados, estados, e das aplicações).
\end{itemize}
	
\subsection{Para a administração do projeto}
\begin{itemize}
	\item Código e documentos disponíveis em repositórios privados Git no BitBucket;
	\item Organização e priorização de tarefas e milestones integrados ao repositório.
\end{itemize}
	
\subsection{Para o desenvolvimento}

\subsubsection{Servidor de comunicação distribuída:}
\begin{itemize}
	\item
		Baseado em \emph{REST}
			\footnote{\emph{Representational State Transfer} - Transferencia de Estado 
				Representativo, modelo de abstracao de comunicacao distribuida baseado
				em hipermidia}
		e desenvolvido em PHP com a biblioteca \emph{Restler}
			\footnote{Biblioteca PHP que abstrai os recursos \emph{REST} para classes de OO};

		\item
		Banco de dados \emph{MySQL} ou \emph{MongoDB}.
			\footnote{MySQL e um banco de dados relacional, enquanto o MongoDB e nao-relacional}
		A decisao dentre um dos dois dependera do formato mais adequado para transporte
		e armazenagem dos dados de fluxo de caixa.
\end{itemize}

\subsubsection{Interface de usuário web:}
\begin{itemize}
	\item
		JavaScript com \emph{framework MV*}
			\footnote{Sigla ``guarda-chuva'' que abriga diversos formatos de organizacao de
				framework originalmente baseados no modelo \emph{MVC}, mas que usam outros
				tipos de componentes para suprir as tarefas de roteamento e controle de
				requisicoes e comportamentos do usuario}
		-- provavelmente o \emph{Ember}, pois ele fornece tecnologias avançadas de interface
		aliadas ao processamento facilitado de armazenamento de dados remotos via \emph{REST};
			
	\item Folhas de estilo \emph{Less}, para melhor manutenção e organização
		\footnote{Less e um pre-processador de CSS que tem como principal funcionalidade a
			possibilidade de aninhar referencias CSS, de modo a diminuir a repeticao de codigo
			e evitar erros, facilitando tambem a manutencao e o entendimento da estrutura};
			
	\item \emph{Layout} fluido e responsivo, para que a utilização do sistema seja agradavel tanto em um
		computador \emph{desktop}, \emph{tablet}, ou celular, mesmo sem um aplicativo móvel dedicado;
		
	\item Capacidade de trabalho \emph{offline} e de sincronização após reconexão, a partir de bibliotecas
		de armazenamento \emph{client-side} \emph{(para em browsers modernos)}.
\end{itemize}

\pagebreak
\subsubsection{Interfaces de usuário móvel:}
Em primeira instância, o aplicativo web suprirá a maioria dos Casos de Uso móveis.
Para além deles, e para a melhoria da \emph{UX}, um aplicativo nativo será criado.
	
\begin{itemize}
\item Plataforma de desenvolvimento único para múltiplos sistemas (que suporte \emph{iOS} e \emph{Android}, e
talvez \emph{Windows~Phone} e \emph{Blackberry}). A plataforma ainda não foi definida, mas o principal cadidato
é o \emph{Appcelerator Titanium}, por permitir o uso de uma só linguagem de desenvolvimento (JS, e o uso opcional
de XML para os templates) e a compilação posterior em aplicativos verdadeiramente nativos -- diferente de outras
plataformas que se baseiam em \emph{WebViews}.	
\end{itemize}




\end{document}